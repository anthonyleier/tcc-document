\chapter{Introdução}
\label{chap:introducao}

A saúde humana sempre foi uma área pilar de toda a sociedade e vem se tornando ainda mais vital para sustentar as demais. Levando em consideração os problemas e situações advindos da pandemia de COVID-19, é necessário pensar em formas de automatizar e auxiliar os profissionais de saúde em suas tarefas, para que consigam focar em problemas mais graves e urgentes. Também com o avanço da tecnologia e dos meios de comunicação, a automação vem se fazendo presente na vida de todos e cada vez mais se torna indispensável nas mais diversas áreas. Para a área da saúde não é diferente, é preciso pensar em formas de, além de automatizar, também facilitar processos cotidianos para assim garantir um foco maior nos problemas mais críticos.

Também como efeito da pandemia, a demanda por exames laboratoriais vem crescendo, e conforme isso acontece, se necessita cada vez mais de profissionais da saúde especializados em atender, analisar e produzir laudos desses exames. Porém nem sempre existe uma equipe suficiente para isso, e então acontece sobrecarga de funções para dar conta dessa demanda.

Esse trabalho tem como principal objetivo buscar maneiras de facilitar e atender a produção de laudos de exames laboratoriais, com um foco em exames de sangue e na produção de hemogramas. De forma que os profissionais da saúde possam utilizar uma ferramenta para auxiliar nesse procedimento. Atualmente, os hemogramas são realizados por máquinas especializadas nessa tarefa e portanto demandam um alto custo financeiro e de manutenção para isso. Esse processo poderia ser facilitado com o uso de algoritmos de \emph{Deep Learning} para a automatização, como forma alternativa ao maquinário especializado.

Os algoritmos de \emph{Deep Learning} (DL) vêm sendo utilizados nas mais diversas áreas, como na medicina \cite{deepLearningMedicine}, na economia \cite{deepLearningEconomy}, nas áreas da educação \cite{deepLearningEducation}, no comércio eletrônico \cite{deepLearningEcommerce} e até em jogos virtuais \cite{deepLearningGaming}. Portanto, DL vem se tornando cada vez mais uma alternativa à métodos tradicionais de realizar tarefas e automatizar processos. Podem ser encontrados alguns trabalhos também na área da saúde, que utilizam técnicas de \emph{Deep Learning} como forma de auxiliar os profissionais em suas tomadas de decisão \cite{deepLearningHealth1} \cite{deepLearningHealth2}.

As técnicas de \emph{Deep Learning} buscam atingir resultados a partir de um grande conjunto de dados. Esses dados devem ser devidamente coletados e adaptados ou seja, pré-processados de forma adequada para a máxima eficiência, dessa forma, um modelo poderá passar por diversas fases de treino, completando o seu treinamento. Com o modelo treinado, pode-se realizar testes com outros dados para obtenção de resultados, que serão pós-processados para uma melhor visualização e apresentados ao profissional da saúde. Todo este processo pode ser chamado de \emph{Knowledge Discovery in Databases} (KDD), que se refere à extração de conhecimento a partir dos dados \cite{kdd} \cite{kdd2}.

Nesse trabalho, busca-se analisar dados de exames de sangue através de imagens de placas de Petri, que são recipientes cilíndricos utilizados pelos profissionais para cultura de microrganismos e análise de materiais \cite{petri}, de forma a elaborar hemogramas e laudos a partir dessas informações. Para isso serão utilizados \emph{datasets} de imagens, a fim de detectar diferentes tipos de células do sangue e chegar em resultados assertivos e úteis para auxiliar também os profissionais da saúde.

% Inicia com uma contextualização, onde se explica como chegou ao problema de pesquisa, e cita-se algumas ideias de trabalhos relacionados. Também deve-se apresentar brevemente a situação atual da área relacionada ao problema que se deseja resolver, conceituar o problema em questão e comentar sobre as técnicas que serão utilizadas para resolver o problema. Tudo isso em no máximo 1 página (3 ou 4 parágrafos).

\section{Problema de Pesquisa}
\label{sec:problema}

Pensando nas formas e aplicações dos algoritmos de \emph{Deep Learning}, presentes nas mais diversas áreas, como um modelo computacional pode ser utilizado para a interpretação de imagens de amostras de sangue em placas de Petri a fim de auxiliar profissionais de laboratório e da saúde na elaboração de laudos científicos e também na sua tomada de decisão?

% Trata-se de uma pergunta, cuja resposta é o seu TCC. Com por exemplo, no TCC do Júlio: “Com base nas técnicas e algoritmos de Machine Learning mais frequentes na Literatura, como ofertar um protótipo com um modelo computacional para predição de evasão escolar a nível de estudante para os cursos de graduação do Instituto Federal de Santa Catarina - Câmpus Caçador?”. Podemos perceber que nosso trabalho é necessariamente uma resposta a uma pergunta. O ideal é que tenha caráter prático, visando a solução de determinado problema.

\section{Hipótese de Pesquisa}
\label{sec:hipotese}

A hipótese para o problema apresentado é que modelos computacionais podem ser treinados para a interpretação de imagens de amostras de sangue em placas de Petri com grande eficiência em prover informações úteis na elaboração automatizada de laudos científicos para profissionais de laboratório e da saúde.

% A hipótese de pesquisa é uma pressuposição sobre o esperado. A observação de uma situação pelo pesquisador, com uma comparação de estudo, dedução lógica da teoria, cultura na qual a problemática é observada, analogias entre duas ou mais variáveis. Normalmente é escrita como uma afirmação associada ao problema de pesquisa. É o que se deseja produzir ao final do TCC.

\section{Objetivos}
\label{sec:objetivos}

\subsection{Objetivo Geral}
Como objetivo geral deste trabalho, deve-se buscar formas de treinamento de um modelo computacional para interpretação de imagens voltado a prover informações úteis sobre hemogramas, possibilitando a geração de laudos científicos automaticamente de forma a auxiliar os profissionais de laboratório e da saúde.

% Converta o seu problema de pesquisa em uma frase afirmativa que inicie com verbo no infinitivo.

\subsection{Objetivos Específicos}
\begin{itemize}
\item Realizar mapeamento sistemático sobre o tema, a fim de identificar as técnicas/algoritmos de \emph{Deep Learning} mais adequados para o reconhecimento de imagens de exames;
\item Buscar dados de imagens de amostras de sangue em bases de dados disponíveis e para esta finalidade;
\item Realizar o pré-processamento dos dados a fim de padronizar e preparar todo o conjunto para o treinamento do modelo computacional;
\item Desenvolver e treinar modelos computacionais de \emph{Deep Learning} a fim de encontrar informações suficientes na análise de amostras de sangue em placas de Petri;
\item Desenvolver um protótipo a partir do modelo computacional pronto e treinado;
\end{itemize}

% Desmembramento do objetivo geral em alguns objetivos específicos, que ao serem atingidos levarão necessariamente ao alcance do objetivo geral.

% Evitar verbos de caráter muito subjetivo, como estudar e conhecer. Preferir utilizar termos como identificar, descrever, propor, entre outros.

\section{Justificativa}
\label{sec:justificativa}
Este estudo busca demonstrar uma forma alternativa de análise das amostras de sangue e na elaboração de laudos, portanto seu principal foco é auxiliar os profissionais da saúde. A contribuição desse estudo poderá ajudar profissionais da saúde a serem mais rápidos em suas decisões sem perder a assertividade, de forma a aumentar a eficiência da análise de exames laboratoriais. Principalmente em momentos de crise, onde a área da saúde é bastante afetada, é necessário ter formas alternativas e associativas em tarefas cotidianas e de extrema importância para a continuidade dos trabalhos. Com esse trabalho, estudiosos da área da computação e também da saúde, poderão ter uma visão muito interessante e associativa de ideias, de forma a auxiliar em novas pesquisas e aplicações.

Outra questão bastante relevante, é em relação aos custos associados, devido ao fato de que o maquinário utilizado hoje para a análise desses exames demanda um custo altíssimo para a sua compra e manutenção. Esse trabalho também possibilitará a análise laboratorial sem a necessidade de compra dessas máquinas caríssimas, de forma a diminuir custos e gastos nesse aspecto.

Embora já existam estudos utilizando \emph{Deep Learning} e também estudos utilizando esses conceitos na área da saúde, esse trabalho tem como principal diferencial trazer a ideia de associar a análise dos modelos de \emph{Deep Learning} com a elaboração de laudos e hemogramas de uma forma automatizada. Logo, se faz necessária a investigação dos conceitos desse trabalho para essa e futuras pesquisas. Este estudo demonstra viabilidade técnica, onde toda a pesquisa e aplicação das definições desse material podem ocorrer durante todo o projeto de trabalho de conclusão de curso. Os livros, artigos e materiais teóricos podem ser providenciados pela instituição e estão disponíveis para o uso.

% Defender a necessidade do estudo, quanto a sua importância, originalidade, oportunidade e viabilidade. Sendo a importância: contribuição do seu estudo na sociedade e na área acadêmica. É importante para quem? Por quê?. A Originalidade: ideia minimamente original. A Oportunidade: período adequado, compatível com as necessidades atuais de conhecimentos e a Viabilidade: existência de recursos necessários para realizar os estudos (tempo, livros, artigos, materiais...)

\section{Organização do texto}
\label{sec:organizacao}

O restante desse trabalho está organizado da seguinte maneira: No \autoref{chap:fund} são apresentados os principais conceitos relacionados a \emph{Deep Learning}, bem como as técnicas estudadas. No \autoref{chap:mapeamento} são apresentados os resultados do mapeamento sistemático da literatura. No \autoref{chap:metodologia} são discutidos os procedimentos metodológicos e no \autoref{chap:cronograma} é apresentado o cronograma para desenvolvimento deste projeto. Por fim, no \autoref{chap:conclusoes} são apresentadas as considerações finais acerca deste trabalho.

% Para finalizar o capítulo introdutório, é interessante que você anuncie ao leitor o conteúdo que ele vai encontrar nos capítulos a seguir. Apresente a informação de que o segundo capítulo é composto pela fundamentação teórica sobre os assuntos X, Y e Z, que o terceiro capítulo trás o mapeamento sistemático, que o quarto capítulo é composto pela metodologia adotada, etc. Normalmente um ou dois parágrafos são suficientes.