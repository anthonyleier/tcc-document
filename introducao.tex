\chapter{Introdução}
\label{chap:introducao}
A saúde humana sempre foi uma área pilar de toda a sociedade e vem se tornando ainda mais vital para sustentar as demais. Considerando os problemas e situações advindos da pandemia de COVID-19, é necessário pensar em formas de automatizar tarefas para agilizar o trabalho dos profissionais de saúde, permitindo que os mesmos tenham foco em problemas que demandam atenção urgente. Também, com o avanço da tecnologia e dos meios de comunicação, a automação vem se fazendo presente na vida de todos, tornando-se indispensável nas mais diversas áreas. Para a área da saúde isso não é diferente, sendo preciso pensar em formas de automatizar processos cotidianos para assim garantir  maior atenção nos problemas mais críticos.

Também como efeito da pandemia, a demanda por exames laboratoriais vêm crescendo, e conforme isso acontece, um maior número de profissionais da saúde são necessários para atender, analisar e produzir laudos desses exames. Porém, nem todo local possui uma equipe adequada para lidar com isso, e então acontece sobrecarga de funções para dar conta dessa demanda.

Este trabalho tem como principal objetivo buscar maneiras de automatizar a produção de hemogramas. Através de detecções automáticas, é possível criar uma ferramenta para auxiliar os profissionais da saúde nessa tarefa. Atualmente, os hemogramas são realizados por máquinas especializadas, portanto, demandam um alto custo financeiro e de manutenção para isso. Esse processo pode ser facilitado com o uso de algoritmos de \emph{Deep Learning} para a automatização, como forma alternativa ao maquinário especializado.

Os algoritmos de \emph{Deep Learning} (DL) vêm sendo utilizados nas mais diversas áreas, como na medicina \cite{deepLearningMedicine}, na economia \cite{deepLearningEconomy}, nas áreas da educação \cite{deepLearningEducation}, no comércio eletrônico \cite{deepLearningEcommerce} e até em jogos virtuais \cite{deepLearningGaming}. Portanto, DL vem se tornando, cada vez mais, uma alternativa aos métodos tradicionais de realizar tarefas e automatizar processos. Podem ser encontrados alguns trabalhos também na área da saúde, que utilizam técnicas de \emph{Deep Learning} como forma de auxiliar os profissionais em suas tomadas de decisão \cite{deepLearningHealth1, deepLearningHealth2}.

As técnicas de \emph{Deep Learning} buscam atingir resultados a partir de um grande conjunto de dados. Esses dados devem ser devidamente coletados e adaptados, ou seja, pré-processados adequadamente para que o processo de aprendizagem do \emph{Deep Learning} possa ser desempenhado da melhor forma possível. O modelo de \emph{Deep Learning} passa pela fase de treinamento e posteriormente por uma fase de realização de testes com outros dados para obtenção de resultados, que serão pós-processados para uma melhor visualização e apresentados ao profissional da saúde.

Nesse trabalho, busca-se analisar imagens de exames de sangue, para uma detecção e contagem automatizada das células da amostra. Normalmente, considerando o processo manual, essas amostras são verificadas e analisadas com o uso de lâminas de vidro e microscópio, onde são realizadas as devidas contagens de células, de forma a elaborar hemogramas a partir dessas informações. Outra abordagem utilizada em laboratórios de maior escala é a utilização de máquinas especializadas para esta contagem, que como já mencionado, envolvem custo financeiro elevado. Neste sentido, este projeto consiste na utilização de um \emph{dataset}, grande conjunto de imagens de amostras de sangue, a fim de detectar e contar os diferentes tipos de células existentes na amostra, de forma a auxiliar os profissionais a elaborarem hemogramas.

\section{Problema de Pesquisa}
\label{sec:problema}

Considerando as diferentes formas e aplicações dos algoritmos de \emph{Deep Learning} presentes nas mais diversas áreas, é possível utilizar um modelo computacional, para a detecção e contagem de células em imagens de amostras de sangue, a fim de gerar informações úteis na elaboração de elementos do hemograma e auxiliar os profissionais de laboratório e da saúde?

\section{Hipótese de Pesquisa}
\label{sec:hipotese}
A hipótese para o problema apresentado, é que modelos computacionais podem ser treinados para a detecção e contagem de células em imagens de amostras de sangue, com grande eficiência em prover informações úteis na elaboração de elementos do hemograma, para auxiliar os profissionais de laboratório e da saúde.

\section{Objetivos}
\label{sec:objetivos}

\subsection{Objetivo Geral}
Como objetivo geral deste trabalho, deve-se buscar formas de treinamento de um modelo computacional para detecção de células em imagens de amostras de sangue, voltado a prover informações úteis para elaborar elementos do hemograma e assim auxiliar os profissionais da área em suas tarefas.

\subsection{Objetivos Específicos}
\begin{itemize}
    \item Realizar um mapeamento sistemático sobre o tema, a fim de reconhecer os melhores trabalhos na literatura sobre as técnicas de \emph{Deep Learning} utilizadas na área da saúde.
    \item Buscar imagens de amostras de sangue em bases de dados disponíveis e voltadas para esta finalidade.
    \item Realizar o pré-processamento dos dados coletados a fim de preparar o conjunto para o treinamento do modelo computacional.
    \item Desenvolver e treinar modelos computacionais de \emph{Deep Learning} a fim de encontrar informações na análise de amostras de sangue;
    \item Desenvolver um protótipo funcional através do modelo computacional, para ser usado por profissionais da área.
\end{itemize}

\section{Justificativa}
\label{sec:justificativa}
Este estudo busca demonstrar uma forma alternativa de análise das amostras de sangue para elaboração de elementos do hemograma, portanto seu principal foco é auxiliar os profissionais da saúde. A contribuição desse estudo poderá ajudar os profissionais a serem mais rápidos em suas decisões sem perder a assertividade, de forma a aumentar a eficiência da análise de exames laboratoriais. Principalmente em momentos de crise, onde a área da saúde é bastante afetada, é necessário ter formas alternativas e associativas em tarefas cotidianas e de extrema importância para a continuidade dos trabalhos. Através dos resultados desse projeto, pesquisadores da área da computação e também da saúde, poderão ter uma noção desse contexto e de suas possibilidades, cultivando novas iniciativas de trabalho, de forma a auxiliar em novas pesquisas e aplicações.

Outra questão bastante relevante, é em relação aos custos associados, devido ao fato de que o maquinário utilizado hoje para a análise desses exames demanda um alto custo para a sua compra e manutenção. Esse trabalho também possibilitará a realização da análise laboratorial reduzindo a necessidade de compra desses equipamentos, e consequentemente diminuindo também os custos envolvidos nesse processo.

Embora já existam estudos utilizando \emph{Deep Learning} e também estudos utilizando esses conceitos na área da saúde, esse trabalho tem a ideia de associar a análise dos modelos de \emph{Deep Learning} com a elaboração de elementos do hemograma de uma forma automatizada. Logo, se faz necessária a investigação dos conceitos desse trabalho para essa e futuras pesquisas.

\section{Organização do texto}
\label{sec:organizacao}
O restante desse trabalho está organizado da seguinte maneira: No \autoref{chap:fund} são apresentados os principais conceitos relacionados a \emph{Deep Learning}, bem como as técnicas estudadas. No \autoref{chap:mapeamento} são apresentados os resultados do mapeamento sistemático da literatura. No \autoref{chap:metodologia} será detalhada toda a metodologia do trabalho, assim como no \autoref{chap:resultados} os resultados da pesquisa. Por fim, no \autoref{chap:conclusoes} são apresentadas as considerações finais acerca deste trabalho.

\section{Código-Fonte}
Todo o código fonte apresentado neste trabalho foi desenvolvido na linguagem Python, utilizando a biblioteca e os recursos do Tensorflow. Os códigos utilizados, assim como alguns processos internos do desenvolvimento foram omitidos e ficaram implícitos na escrita final para não se afastar do propósito principal, porém todos os detalhes e aprofundamentos em código do processo podem ser acessados no repositório do Github. \cite{anthonyGithub}
