\chapter{Introdução}
\label{chap:introducao}

A saúde humana sempre foi uma área pilar de toda a sociedade e hoje em dia ela se tornou ainda mais vital para sustentar as demais. Com os problemas e situações advindos da pandemia, devemos pensar em formas de automatizar e auxiliar os profissionais de saúde para que consigam focar em problemas mais graves e urgentes. Com o avanço da tecnologia e dos meios de comunicação, a automação vem se fazendo presente na vida de todos e cada vez mais se torna indispensável nas mais diversas áreas. Para a área da saúde, devemos pensar em formas de além de automatizar, também facilitar processos cotidianos para assim garantir um foco maior nos problemas mais críticos.

A demanda por exames laboratoriais vem crescendo, e conforme isso acontece, se necessita cada vez mais de profissionais da saúde especializados em atender, analisar e produzir laudos desses exames. Porém nem sempre existe uma equipe suficiente formada para isso, e então acontece sobrecarregamentos de funções para dar conta dessa demanda.

Esse trabalho tem como principal objetivo, buscar formas de facilitar e atender a produção de laudos de exames laboratoriais, com um foco em exames de sangue e na produção de hemogramas. De forma, que os profissionais da saúde possam utilizar uma ferramenta para auxiliar nesse procedimento. Hoje em dia, os hemogramas são feitos por máquinas especializadas nessa tarefa e portanto demandam um alto custo financeiro e de manutenção para isso. Esse processo poderia ser facilitado com o uso de algoritmos de deep learning para a automatização como forma alternativa ao maquinário especializado.

Algoritmos de deep learning têm sido empregados nas mais diversas áreas, como na medicina, na indústria de alimentos, no mercado financeiro, na educação, no comercio eletrônico, nos jogos virtuais.

Podemos encontrar muitos trabalhos utilizando técnicas de deep learning com o objetivo de auxiliar profissionais da saúde em tomadas de decisão.

% Inicia com uma contextualização, onde se explica como chegou ao problema de pesquisa, e cita-se algumas ideias de trabalhos relacionados. Também se deve apresentar brevemente a situação atual da área relacionada ao problema que se deseja resolver, conceituar o problema em questão e comentar sobre as técnicas que serão utilizadas para resolver o problema. Tudo isso em no máximo 1 página (3 ou 4 parágrafos).

\section{Problema de Pesquisa}
\label{sec:problema}

Pensando nas formas e utilizações dos algoritmos de Deep Learning, utilizados nas mais diversas áreas, como podes utilizar um modelo computacional para a interpretação de imagens de sangue a fim de auxiliar profissionais da saúde na sua tomada de decisão e na elaboração de laudos científicos?

% Trata-se de uma pergunta, cuja resposta é o seu TCC. Com por exemplo, no TCC do Júlio: “Com base nas técnicas e algoritmos de Machine Learning mais frequentes na Literatura, como ofertar um protótipo com 
% um modelo computacional para predição de evasão escolar a nível de estudante para os cursos de graduação do Instituto Federal de Santa Catarina - Câmpus Caçador?”. Podemos perceber que nosso trabalho é necessariamente uma resposta a uma pergunta. O ideal é que tenha caráter prático, visando a solução de determinado problema.

\section{Hipótese de Pesquisa}
\label{sec:hipotese}
A hipótese para a solução do problema apresentado é que podemos utilizar modelos treinado para a interpretação de imagens de sangue com grandes capacidades de prover informações uteis na tomada de decisão dos profissionais da saúde.

% A hipótese de pesquisa é uma pressuposição sobre o esperado. A observação de uma situação pelo pesquisador, com uma comparação de estudo, dedução lógica da teoria, cultura na qual a problemática é observada, analogias entre duas ou mais variáveis. Normalmente é escrita como uma afirmação associada ao problema de pesquisa. É o que se deseja produzir ao final do TCC.

\section{Objetivos}
\label{sec:objetivos}

\subsection{Objetivo Geral}
Como objetivo geral deste trabalho, se deve buscar formas de treinamento de um modelo para interpretação de imagens voltado a prover informações úteis na tomada de decisão dos profissionais da saúde.

% Converta o seu problema de pesquisa em uma frase afirmativa que inicie com verbo no infinitivo.

\subsection{Objetivos Específicos}
\begin{itemize}
   \item Realizar mapeamento sistemático sobre o tema, a fim de identificar as técnicas/algoritmos de
Deep Learning mais frequentes e recentes para auxiliar os profissionais.
   \item Buscar dados de imagens em bases de dados disponíveis e para esse fim.
   \item Realizar o pré-processamento dos dados a fim de padronizar e preparar todo o conjunto para o treinamento do modelo.
   \item Desenvolver e treinar modelos computacionais de Deep Learning a fim de encontrar informações suficientes para a análise do sangue.
   \item Desenvolver um protótipo a partir do modelo pronto e treinado.
\end{itemize}

% Desmembramento do objetivo geral em alguns objetivos específicos, que ao serem atingidos levarão necessariamente ao alcance do objetivo geral.

% Evitar: verbos de caráter muito subjetivo (estudar, 
% conhecer...); preferir (identificar, descrever, propor...)

\section{Justificativa}
\label{sec:justificativa}
Este estudo busca demonstrar uma forma alternativa de análise de sangue, como o uso de deep learning está cada vez mais sendo usado nas mais diversas áreas. A motivação para este trabalho partiu de pesquisas semelhantes em várias instituições no Brasil e no mundo. A metodologia abordada nesta pesquisa é a de KDD. Todo o resultado obtido neste trabalho pode servir de base para novos projetos que surgirem a respeito deste tema, podendo ser reaproveitado o dataset, ou o pré-processamento, os atributos escolhidos. Além disso, os resultados podem ser utilizados em comparações em testes de novos algoritmos.

% Defender a necessidade do estudo, quanto a sua importância, originalidade, oportunidade e viabilidade. Sendo a importância: contribuição do seu estudo na sociedade e na área acadêmica. É importante para 
% quem? Por quê?. A Originalidade: ideia minimamente original. A Oportunidade: período adequado, compatível com as necessidades atuais de conhecimentos e a Viabilidade: existência de recursos necessários para 
% realizar os estudos (tempo, livros, artigos, materiais...)

\section{Organização do texto}
\label{sec:organizacao}

Organização do Texto.

% O restante deste texto está organizado da seguinte forma: No \autoref{chap:fund} são apresentados os principais conceitos relacionados a <assunto estudado>, bem como as técnicas <X, Y e Z> estudadas. No \autoref{chap:mapeamento} são apresentados os resultados do mapeamento sistemático da literatura. No \autoref{chap:metodologia} são discutidos os procedimentos metodológicos e no \autoref{chap:cronograma} é apresentado o cronograma para desenvolvimento deste projeto. Por fim, no \autoref{chap:conclusoes} são apresentadas as considerações finais acerca deste trabalho.

% Para finalizar o capítulo introdutório, é interessante que você anuncie ao leitor o conteúdo que ele vai encontrar nos capítulos a seguir. Apresente a informação de que o segundo capítulo é composto pela fundamentação teórica sobre os assuntos X, Y e Z, que o terceiro capítulo trás o mapeamento sistemático, que o quarto capítulo é composto pela metodologia adotada, etc. Normalmente um ou dois parágrafos são suficientes.