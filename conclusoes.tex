\chapter{Conclusões}
\label{chap:conclusoes}

Este projeto iniciou com o interesse de utilizar as técnicas de \emph{Deep Learning} para auxiliar os profissionais da saúde em seus trabalhos, pois com a automatização que essa técnica oferece, outras tarefas podem ter maior atenção. O hemograma é o exame mais básico, mais essencial e também o mais realizado. Portanto, os estudos subsequentes foram direcionados a estudar quais seriam as melhores abordagens para a automatização dessa tarefa.

Foi possível entender todo o processo de produção de hemograma e também encontrar as formas de associar os conceitos de \emph{Deep Learning} nesse processo. Através do treinamento de um modelo computacional inteligente capaz de reconhecer as diferentes células do sangue, desenvolveu-se a contagem de todas os tipos de célula com grande precisão.

Essa abordagem desenvolvida é de grande utilidade, pois permite tirar conclusões sobre o estado atual da saúde do paciente. A contagem das células brancas, vermelhas e também das plaquetas, permitem avaliar se existe algum desequilíbrio em seus níveis, caso apresente algum fator anormal, outros exames mais aprofundados podem ser realizados para verificar estas questões. Entretanto, não se caracteriza como um hemograma completo, pois devido a limitações do \emph{dataset} utilizado, o trabalho também apresentou limitações, principalmente relacionado a classificação das células brancas, onde a contagem foi geral.

Para estudos futuros, é possível utilizar os conceitos aqui abordados para aplicar estes conhecimentos em um conjunto de dados diferente e mais complexo, ou também voltado para outro fim. Outra possibilidade é utilizar métricas de avaliação mais voltadas a essa prática, considerando a área das predições e intersecções com os valores reais. Com poucas adaptações é possível continuar a abordagem adotada por este trabalho para contribuir na literatura, calculando a contagem em relação ao volume e posteriormente buscando reproduzir todas as informações presentes em um hemograma.
