\chapter{Fundamentação Teórica}
\label{chap:fund}

Neste capítulo serão apresentados os principais tópicos relacionados ao <Assunto Estudado>, seu conceito e seus impactos na sociedade, bem como as motivações para suas publicações e formas de identificá-las. Além disso, serão abordadas técnicas que permitem <Descrever as técnicas utilizadas>, que serão aplicados para <Tema Proposto>. 

\section{Conceito 1}
\label{sec:conceito1}

Abaixo é apresentada uma figura com o logotipo do Instituto Federal de Santa Catarina. Para inserir uma figura usando o LaTeX, utilizamos a diretiva \emph{figure}. Normalmente referenciamos a figura a partir do seu label, conforme segue. A Figura \ref{fig:exemplo1} mostra o exemplo de uso de imagenos no \LaTeX.

\begin{figure}[!htb]
    \centering
    \caption{Exemplo de uso de imagens no \LaTeX.}
    \includegraphics[width=0.40\textwidth]{img/ifsc.png}
    \legend{Fonte: Elaborada pelo autor.}
    \label{fig:exemplo1}
 \end{figure}
 
 Observe todos os detalhes utilizados. A diretiva \emph{centering} é utilizada para deixar a imagem centralizada. A diretiva \emph{caption} é utilizada para adicionar a legenda na parte superior da imagem. A diretiva \emph{includegraphics} serve para adicionar a imagem propriamente dita, estando neste caso, localizada dentro da pasta \emph{img}. Na mesma diretiva, é possível notar o código \texttt{width=0.40}, que significa que a imagem vai utilizar 40\% da largura do texto. Por fim, a diretiva \emph{legend} é utilizada para indicar a fonte da imagem, e a diretiva \emph{label} para criar uma referência.

\section{Conceito 2}
\label{sec:conceito2}

\section{Conceito 3}
\label{sec:conceito3}

\chapter{Estado da Arte da Área Pesquisada}
\label{chap:mapeamento}

O processo de pesquisa e seleção dos trabalhos relacionados, foi realizado com base em um mapeamento sistemático sobre as pesquisas com propostas para agilizar a identificação e interpretação de análises de sangue. Esta revisão resultou na identificação e seleção dos principais trabalhos de pesquisa no tema deste Projeto de Trabalho de Conclusão de Curso. Outro objetivo deste mapeamento sistemático foi verificar os métodos utilizados para a aplicação de Deep Learning em imagens de sangue em placas de petri de maneira que possam ser aplicados neste projeto de forma satisfatória.

\section{Mapeamento Sistemático da Literatura}

O mapeamento sistemático da literatura é realizado com base na busca e levantamento de artigos, para isso se utiliza uma string de busca para as principais bibliotecas e repositórios de artigos. Esses artigos serão analisados e selecionados conforme a sua área de pesquisa e a sua temática, para inclusão nesse estudo. Para isso, se é utilizado uma ferramenta para automatização dessa tarefa, que é o Parsifal\footnote[1]{https://parsif.al/}, de modo a definir a string de busca, salvar os artigos necessários e realizar a seleção.

As questões de pesquisas levantadas para isso foram, ``Como os algoritmos de Deep Learning podem ser utilizados para a interpretação de exames?'' e ``Como realizar o tratamento de imagens para reconhecimento por modelos de Deep Learning?''. A partir dessas questões se foram extraídas palavras e termos para o direcionamento da pesquisa. Podemos visualizar estas palavras com seus sinônimos na Tabela 1.

\begin{center}
Tabela 1 - Tabela com Palavras-Chave e Sinônimos
\begin{center}
\begin{tabular}{|c|c|}
\hline
\textbf{Palavra-Chave} & \textbf{Sinônimos} \\ \hline
Blood Analysis & Blood Sample \\ \hline
Classification & Interpretation, Recognition \\ \hline
Deep Learning & Artificial Intelligence, Computer Vision, Machine Learning \\ \hline
\end{tabular}
\end{center}
Fonte: Elaborada pelo Autor
\end{center}

Na Tabela 2, é listado as bases de dados onde os artigos foram coletados, a quantidade de cada um de les e a string de busca utilizada na seleção. A mesma string de busca foi utilizado nas três bases de dados, e os artigos encontrados foram dos últimos 5 anos.

\clearpage
\begin{center}
Tabela 2 - Bases de Dados e Quantidade de Artigos Selecionados
\begin{center}
\begin{tabular}{|c|c|c|}
\hline
\textbf{Base de Dados} & \textbf{Artigos} & \textbf{String de Busca} \\ \hline
\multirow{2}{*}{ACM Digital Library} & \multirow{2}{*}{37} & \multirow{6}{*}{\begin{tabular}[c]{@{}c@{}}(``classification'' OR ``interpretation'' OR ``recognition'') AND\\  (``deep learning'' OR ``artificial intelligence'' \\ OR ``computer vision'' OR ``machine learning'') AND\\  (``blood analysis'' OR ``blood sample'')\end{tabular}} \\
 &  &  \\ \cline{1-2}
\multirow{2}{*}{IEEE Digital Library} & \multirow{2}{*}{13} &  \\
 &  &  \\ \cline{1-2}
\multirow{2}{*}{Scopus} & \multirow{2}{*}{114} &  \\
 &  &  \\ \hline
\end{tabular}
\end{center}
Fonte: Elaborada pelo Autor
\end{center}

\subsection{Critérios de Exclusão}

Os artigos coletados na pesquisa através da string de busca, passaram por critérios de exclusão por não se adequarem a esta pesquisa, esses critérios podem ser observados na Tabela 3. 

\begin{center}
Tabela 3 - Critérios de Exclusão
\begin{center}
\begin{tabular}{|c|c|}
\hline
\textbf{Critério de Exclusão} & \textbf{Nº de Artigos Recusados} \\ \hline
O estudo não faz parte da área de pesquisa & 101 \\ \hline
O estudo apresenta resultados fora da computação & 29 \\ \hline
O estudo não é um estudo primário & 6 \\ \hline
O estudo é duplicado & 16 \\ \hline
\end{tabular}
\end{center}
Fonte: Elaborada pelo Autor
\end{center}

A seleção inciou com 164 artigos no total das três bases de dados buscadas. Com a aplicação dos critérios de exclusão, observa-se um resultante de apenas 14 artigos. Isso ocorreu pois 101 artigos foram eliminados no critério ``O estudo não faz parte da área de pesquisa'', que significa que esses artigos tinham alguma relação, porém eram voltados a outras áreas. Outros 29 artigos foram eliminados no critério ``O estudo apresenta resultados fora da computação'', que significa que eram da área de pesquisa, porém com resultados e métodos sem conexão com a computação. Foram também encontrados 6 artigos, que entraram no critério ``O estudo não é um estudo primário'', o que indica que o artigo pode ser uma revisão sistemática da literatura ou semelhante. Por fim, foram eliminados outros 16 artigos por serem duplicados.

\subsection{Critérios de Inclusão}

Os seguintes critérios de inclusão foram definidos:
\begin{itemize}
\item Nova tecnologia para análise de sangue;
\item Processo, método ou técnica para contagem de células sanguíneas;
\item Sistema para elaboração de hemogramas utilizando Deep Learning;
\end{itemize}

Na tabela 4, podemos encontrar todos os 14 artigos selecionados com base nos critérios de inclusão, todos eles se enquadram em pelo menos um deles.

\begin{center}
Tabela 4 - Artigos Selecionados
\begin{center}
\begin{tabular}{|c|l|l|}
\hline
\textbf{ID} & \multicolumn{1}{c|}{\textbf{Título do Artigo}} & \multicolumn{1}{c|}{\textbf{Autores}} \\ \hline
A1 & \begin{tabular}[c]{@{}l@{}}Analyzing microscopic images of \\ peripheral blood smear \\ using deep learning\end{tabular} & \begin{tabular}[c]{@{}l@{}}Mundhra, D. and Cheluvaraju, B. \\ and Rampure, J. and Rai Dastidar, T.\end{tabular} \\ \hline
A2 & \begin{tabular}[c]{@{}l@{}}Automatic detection and classification \\ of leukocytes using \\ convolutional neural networks\end{tabular} & \begin{tabular}[c]{@{}l@{}}Zhao, J. and Zhang, M. \\ and Zhou, Z. and Chu, J. and Cao, F.\end{tabular} \\ \hline
A3 & \begin{tabular}[c]{@{}l@{}}Automatic white blood cell classification \\ using pre-trained deep learning models: \\ ResNet and Inception\end{tabular} & \begin{tabular}[c]{@{}l@{}}Habibzadeh, M. and Jannesari, M. \\ and Rezaei, Z. and Baharvand, H. \\ and Totonchi, M.\end{tabular} \\ \hline
A4 & \begin{tabular}[c]{@{}l@{}}Classification of Human White \\ Blood Cells Using Machine Learning \\ for Stain-Free Imaging \\ Flow Cytometry\end{tabular} & \begin{tabular}[c]{@{}l@{}}Lippeveld, M. and Knill, C. and \\ Ladlow, E. and \\ Fuller, A. and Michaelis, L.J. and \\ Saeys, Y. and Filby, A. and Peralta, D.\end{tabular} \\ \hline
A5 & \begin{tabular}[c]{@{}l@{}}Blood cell classification using the hough\\ transform and \\ convolutional neural networks\end{tabular} & \begin{tabular}[c]{@{}l@{}}Molina-Cabello, M.A. and López-Rubio, E. \\ and Luque-Baena, R.M. and \\ Rodríguez-Espinosa, M.J. and \\ Thurnhofer-Hemsi, K.\end{tabular} \\ \hline
A6 & \begin{tabular}[c]{@{}l@{}}White Blood Cells Image Classification \\ Using Deep Learning with \\ Canonical Correlation Analysis\end{tabular} & Patil, A.M. and Patil, M.D. and Birajdar, G.K. \\ \hline
A7 & \begin{tabular}[c]{@{}l@{}}Image processing and machine learning\\ in the morphological analysis \\ of blood cells\end{tabular} & \begin{tabular}[c]{@{}l@{}}Rodellar, J. and Alférez, S. and Acevedo, A. \\ and Molina, A. and Merino, A.\end{tabular} \\ \hline
A8 & \begin{tabular}[c]{@{}l@{}}Improving blood cells classification in \\ peripheral blood smears using \\ enhanced incremental training\end{tabular} & Al-qudah, R. and Suen, C.Y. \\ \hline
A9 & \begin{tabular}[c]{@{}l@{}}Corruption-Robust Enhancement of \\ Deep Neural Networks\\ for Classification of Peripheral \\ Blood Smear Images\end{tabular} & \begin{tabular}[c]{@{}l@{}}Zhang, S. and Ni, Q. and Li, B. and \\ Jiang, S. and \\ Cai, W. and Chen, H. and Luo, L.\end{tabular} \\ \hline
A10 & \begin{tabular}[c]{@{}l@{}}Convolutional neural network and decision \\ support in medical imaging:\\ Case study of the recognition of \\ blood cell subtypes\end{tabular} & Diouf, D. and Seck, D. and Diop, M. and Ba, A. \\ \hline
A11 & \begin{tabular}[c]{@{}l@{}}Combining Convolutional Neural Network\\ With Recursive Neural Network \\ for Blood Cell Image Classification\end{tabular} & \begin{tabular}[c]{@{}l@{}}Liang, G. and Hong, H. and Xie, W. and\\ Zheng, L.\end{tabular} \\ \hline
A12 & \begin{tabular}[c]{@{}l@{}}Blood diseases detection using \\ classical machine learning algorithms\end{tabular} & Alsheref, F.K. and Gomaa, W.H. \\ \hline
\end{tabular}
\end{center}
Fonte: Elaborada pelo Autor
\end{center}

Todos os artigos selecionados estão relacionados à maneiras e recursos para auxiliar na interpretação de exames de sangue utilizando conceitos de Deep Learning e Machine Learning.

\section{Análise dos trabalhos selecionados}

Por fim, com os artigos selecionados e classificados, é necessário realizar a extração dos dados desses trabalhos, sendo essa a última etapa desse mapeamento sistemático da literatura. É possível perceber que os algoritmos e abordagens mais utilizados são técnicas de \emph{Deep Learning}, como por exemplo, o uso de \emph{Convolutional Neural Network (CNN)} (A1, A2, A3, A4, A5, A6, A8, A9, A10, A11) e de \emph{Recurrent Neural Network (RNN)} (A6, A11), que são abordagens de redes neurais para a classificação das células sanguíneas.

Outros trabalhos utilizam de algoritmos de \emph{Machine Learning} tradicionais para a classificação, como por exemplo, ocorre com o uso de \emph{Random Forest} ou \emph{Decision Trees}  (A2, A4, A7), que são estruturas de árvores de decisão. Também se encontra estudos fazendo uso de \emph{Support Vector Machine (SVM)} (A7) que utilizam vetores de suporte e por fim \emph{K-Means e K-Nearest Neighbors (KNN)} (A12), que faz a classificação levando em conta os vizinhos mais próximos.

\chapter{Procedimentos Metodológicos}
\label{chap:metodologia}

\section{Recursos}

\chapter{Cronograma}
\label{chap:cronograma}

A Tabela \ref{tbl:cronograma} apresenta o cronograma de atividades propostas para o desenvolvimento deste projeto de trabalho de conclusão de curso, de forma a viabilizar <Falar sobre o que se pretende atingir com o projeto>.

\begin{table}[!htb]
\centering
\caption{Cronograma das atividades previstas.}
\label{tbl:cronograma}
\begin{tabular}{|l|c|c|c|c|c|c|c|c|c|c|}
\hline
\multicolumn{1}{|c|}{\textbf{Etapa}}       & \multicolumn{10}{c|}{\textbf{Meses}}                                                                                                                        \\ \hline
                                           & Fev & Mar & Abr & Mai & Jun & Ago & Set & Out & Nov & Dez \\ \hline
Fundamentação Teórica                      & X            & X            &              &              &              &                &                   &               &              &              \\ \hline
\makecell[l]{Mapeamento Sistemático \\ da Literatura}       &              &              & X            & X            &              &                &                   &               &              &              \\ \hline
\makecell[l]{Escrita do Projeto de TCC \\ e Defesa}         &              &              & X            & X            & X            &                &                   &               &              &              \\ \hline
\makecell[l]{Atividade a ser desenvolvida 1}              &              &              &              &              &              & X              &                   &               &              &              \\ \hline
\makecell[l]{Atividade a ser desenvolvida 2}             &              &              &              &              &              &                & X                 &               &              &              \\ \hline
\makecell[l]{Atividade a ser desenvolvida 3} &              &              &              &              &              &                & X                 & X             &              &              \\ \hline
\makecell[l]{Verificação de Aceitação dos \\ Resultados}    &              &              &              &              &              &                &                   & X             &              &              \\ \hline
\makecell[l]{Comparação dos Resultados \\ com a Literatura} &              &              &              &              &              &                &                   & X             & X            &              \\ \hline
Exposição dos Resultados                   &              &              &              &              &              &                &                   &               & X            &              \\ \hline
Escrita do TCC                             &              &              &              &              &              &                &                   &               & X            & X            \\ \hline
Defesa do TCC                              &              &              &              &              &              &                &                   &               &              & X            \\ \hline
\end{tabular}
\vspace{6pt}
\legend{Fonte: Elaborada pelo autor.}
\end{table}

As atividades propostas neste cronograma podem sofrer leves alterações no decorrer do seu desenvolvimento de acordo com a necessidade.

A forma mais fácil de criar tabelas é através de ferramentas gráficas. Geralmente utiliza-se o site \url{https://www.tablesgenerator.com/} para realizar tal atividade, exportando o código LaTeX e colando na parte do texto que ela deve aparecer~\cite{tablegenerator2021}.