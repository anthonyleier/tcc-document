\chapter{Fundamentação Teórica}
\label{chap:fund}

Neste capítulo serão apresentados os principais tópicos relacionados ao <Assunto Estudado>, seu conceito e seus impactos na sociedade, bem como as motivações para suas publicações e formas de identificá-las. Além disso, serão abordadas técnicas que permitem <Descrever as técnicas utilizadas>, que serão aplicados para <Tema Proposto>. 

\section{Conceito 1}
\label{sec:conceito1}

Abaixo é apresentada uma figura com o logotipo do Instituto Federal de Santa Catarina. Para inserir uma figura usando o LaTeX, utilizamos a diretiva \emph{figure}. Normalmente referenciamos a figura a partir do seu label, conforme segue. A Figura \ref{fig:exemplo1} mostra o exemplo de uso de imagenos no \LaTeX.

\begin{figure}[!htb]
    \centering
    \caption{Exemplo de uso de imagens no \LaTeX.}
    \includegraphics[width=0.40\textwidth]{img/ifsc.png}
    \legend{Fonte: Elaborada pelo autor.}
    \label{fig:exemplo1}
 \end{figure}
 
 Observe todos os detalhes utilizados. A diretiva \emph{centering} é utilizada para deixar a imagem centralizada. A diretiva \emph{caption} é utilizada para adicionar a legenda na parte superior da imagem. A diretiva \emph{includegraphics} serve para adicionar a imagem propriamente dita, estando neste caso, localizada dentro da pasta \emph{img}. Na mesma diretiva, é possível notar o código \texttt{width=0.40}, que significa que a imagem vai utilizar 40\% da largura do texto. Por fim, a diretiva \emph{legend} é utilizada para indicar a fonte da imagem, e a diretiva \emph{label} para criar uma referência.

\section{Conceito 2}
\label{sec:conceito2}

\section{Conceito 3}
\label{sec:conceito3}

\chapter{Estado da Arte da Área Pesquisada}
\label{chap:mapeamento}

O processo de pesquisa e seleção dos trabalhos relacionados, foi realizado com base em um mapeamento sistemático sobre as pesquisas com propostas para agilizar a identificação e interpretação de análises de sangue. Esta revisão resultou na identificação e seleção dos principais trabalhos de pesquisa no tema deste Projeto de Trabalho de Conclusão de Curso. Outro objetivo deste mapeamento sistemático foi verificar os métodos utilizados para a aplicação de Deep Learning em imagens de sangue em placas de petri de maneira que possam ser aplicados neste projeto de forma satisfatória.

\section{Mapeamento Sistemático da Literatura}

O mapeamento sistemático da literatura é realizado com base na busca e levantamento de artigos, para isso se utiliza uma string de busca para as principais bibliotecas e repositórios de artigos. Esses artigos serão analisados e selecionados conforme a sua área de pesquisa e a sua temática, para inclusão nesse estudo. Para isso, se é utilizado uma ferramenta para automatização dessa tarefa, que é o Parsifal\footnote[1]{https://parsif.al/}, de modo a definir a string de busca, salvar os artigos necessários e realizar a seleção.

As questões de pesquisas levantadas para isso foram, "Como os algoritmos de Deep Learning podem ser utilizados para a interpretação de exames?" e "Como realizar o tratamento de imagens para reconhecimento por modelos de Deep Learning?". A partir dessas questões se foram extraídas palavras e termos para o direcionamento da pesquisa. Podemos visualizar estas palavras com seus sinônimos na Tabela 1.

\begin{table}[]
\begin{tabular}{|l|l|}
\hline
\textbf{Palavra-Chave} & \textbf{Sinônimos} \\ \hline
Blood Analysis & Blood Sample \\ \hline
Classification & Interpretation, Recognition \\ \hline
Deep Learning & Artificial Intelligence, Computer Vision, Machine Learning \\ \hline
\end{tabular}
\end{table}

Na Tabela 2, são listadas as bases de dados em que foram pesquisados os artigos juntamente com
a string de busca utilizada e o número de artigos que foram retornados com esta busca. Como pode ser
notado, a mesma string de busca foi utilizada para as três bases de dados.

\begin{table}[]
\begin{tabular}{|c|c|c|}
\hline
\textbf{Base de Dados} & \textbf{Artigos} & \textbf{String de Busca} \\ \hline
ACM Digital Library & 37 & \multirow{3}{*}{\begin{tabular}[c]{@{}c@{}}("classification" OR "interpretation" OR "recognition") AND\\  ("deep learning" OR "artificial intelligence" OR "computer vision" OR "machine learning") AND\\  ("blood analysis" OR "blood sample")("classification" OR "interpretation" OR "recognition") AND\\  ("deep learning" OR "artificial intelligence" OR "computer vision" OR "machine learning") AND\\  ("blood analysis" OR "blood sample")\end{tabular}} \\ \cline{1-2}
IEEE Digital Library & 13 &  \\ \cline{1-2}
Scopus & 114 &  \\ \hline
\end{tabular}
\end{table}

\subsection{Critérios de Exclusão}



\subsection{Critérios de Inclusão}


\section{Análise dos trabalhos selecionados}

\chapter{Procedimentos Metodológicos}
\label{chap:metodologia}

\section{Recursos}

\chapter{Cronograma}
\label{chap:cronograma}

A Tabela \ref{tbl:cronograma} apresenta o cronograma de atividades propostas para o desenvolvimento deste projeto de trabalho de conclusão de curso, de forma a viabilizar <Falar sobre o que se pretende atingir com o projeto>.

\begin{table}[!htb]
\centering
\caption{Cronograma das atividades previstas.}
\label{tbl:cronograma}
\begin{tabular}{|l|c|c|c|c|c|c|c|c|c|c|}
\hline
\multicolumn{1}{|c|}{\textbf{Etapa}}       & \multicolumn{10}{c|}{\textbf{Meses}}                                                                                                                        \\ \hline
                                           & Fev & Mar & Abr & Mai & Jun & Ago & Set & Out & Nov & Dez \\ \hline
Fundamentação Teórica                      & X            & X            &              &              &              &                &                   &               &              &              \\ \hline
\makecell[l]{Mapeamento Sistemático \\ da Literatura}       &              &              & X            & X            &              &                &                   &               &              &              \\ \hline
\makecell[l]{Escrita do Projeto de TCC \\ e Defesa}         &              &              & X            & X            & X            &                &                   &               &              &              \\ \hline
\makecell[l]{Atividade a ser desenvolvida 1}              &              &              &              &              &              & X              &                   &               &              &              \\ \hline
\makecell[l]{Atividade a ser desenvolvida 2}             &              &              &              &              &              &                & X                 &               &              &              \\ \hline
\makecell[l]{Atividade a ser desenvolvida 3} &              &              &              &              &              &                & X                 & X             &              &              \\ \hline
\makecell[l]{Verificação de Aceitação dos \\ Resultados}    &              &              &              &              &              &                &                   & X             &              &              \\ \hline
\makecell[l]{Comparação dos Resultados \\ com a Literatura} &              &              &              &              &              &                &                   & X             & X            &              \\ \hline
Exposição dos Resultados                   &              &              &              &              &              &                &                   &               & X            &              \\ \hline
Escrita do TCC                             &              &              &              &              &              &                &                   &               & X            & X            \\ \hline
Defesa do TCC                              &              &              &              &              &              &                &                   &               &              & X            \\ \hline
\end{tabular}
\vspace{6pt}
\legend{Fonte: Elaborada pelo autor.}
\end{table}

As atividades propostas neste cronograma podem sofrer leves alterações no decorrer do seu desenvolvimento de acordo com a necessidade.

A forma mais fácil de criar tabelas é através de ferramentas gráficas. Geralmente utiliza-se o site \url{https://www.tablesgenerator.com/} para realizar tal atividade, exportando o código LaTeX e colando na parte do texto que ela deve aparecer~\cite{tablegenerator2021}.