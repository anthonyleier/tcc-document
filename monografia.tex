%------------------------------------------------------------------------%
% ABNTEX2: Modelo de Trabalho Acadêmico (tese de doutorado, dissertação de mestrado e trabalhos monográficos em geral) em conformidade com ABNT NBR 14724:2011
% Adaptado por Samuel da Silva Feitosa (2021-02-18) baseado no modelo "Template para elaboração de trabalho acadêmico" disponível na página de documentos úteis do IFSC
%------------------------------------------------------------------------%

\documentclass[
	% -- opções da classe memoir -- %
	10pt,			 	  % tamanho da fonte
	%openright,			  % capítulos começam em página ímpar (insere página vazia caso preciso)
	%twoside,			  % para impressão em frente e verso, ou seja, oposto a oneside
	oneside,
	a4paper,			  % tamanho do papel. 
	% -- opções da classe abntex2 -- %
	chapter=TITLE,		  % títulos de capítulos convertidos em letras maiúsculas
	%section=TITLE,		  % títulos de seções convertidos em letras maiúsculas
	%subsection=TITLE,	  % títulos de subseções convertidos em letras maiúsculas
	%subsubsection=TITLE, % títulos de subsubseções convertidos em letras maiúsculas
	% -- opções do pacote babel -- %
	english,			 % idioma adicional para hifenização
%	french,				 % idioma adicional para hifenização
%	spanish,			 % idioma adicional para hifenização
	brazil				 % o último idioma é o principal do documento
	]{abntex2}

% Todas as indicações de pacotes e configurações estão no arquivo de estilo chamado estilo-monografia-ifsc.sty.
\usepackage{style}	
	
%---------------------------------------------------------------------%
% Informações de dados para Capa e Folha de Rosto
%---------------------------------------------------------------------%

\titulo{APLICANDO DEEP LEARNING EM EXAMES LABORATORIAIS DE SANGUE \\ Desenvolvimento de Laudos e Hemogramas}
\autor{Anthony Cruz}
\local{Caçador - SC}
\data{24 de Março de 2021}
\orientador{Professor Samuel da Silva Feitosa}
\coorientador{Professor Cristiano Mesquita Garcia}
\instituicao{%
  Instituto Federal de Santa Catarina -- IFSC
  \par
  Campus Caçador
  \par
  Sistemas de Informação}
\tipotrabalho{Monografia (Graduação)}

% O preambulo deve conter o tipo do trabalho, o objetivo, o nome da instituição e a área de concentração
\preambulo{Projeto de Pesquisa apresentado à Coordenadoria do Curso de Sistemas de Informação do Câmpus Caçador do Instituto Federal de Santa Catarina para a avaliar a possibilidade de continuidade do Trabalho de Conclusão de Curso.}
%---------------------------------------------------------------------%

\textoaprovacao{Este projeto foi julgado adequado para continuidade do Trabalho de Conclusão do Curso de Sistemas de Informação, pelo Instituto Federal de Educação, Ciência e Tecnologia de Santa Catarina, e aprovado na sua forma final pela comissão avaliadora abaixo indicada.}


%---------------------------------------------------------------------%
% Início do Documento
%---------------------------------------------------------------------%
\begin{document}
% Seleciona o idioma do documento (conforme pacotes do babel)
\selectlanguage{brazil}
% Retira espaço extra obsoleto entre as frases.
\frenchspacing 

% ----------------------------------------------------------
% Elementos Pré-Textuais
% ----------------------------------------------------------
% \pretextual
\imprimircapa

% Folha de Rosto - (o * indica que haverá a ficha bibliográfica)
\imprimirfolhaderosto
%---------------------------------------------------------------------%

% Folha de aprovação

% Isto é um exemplo de Folha de aprovação, elemento obrigatório da NBR 14724/2011 (seção 4.2.1.3). Você pode utilizar este modelo até a aprovação do trabalho. 
% Após isso, substitua todo o conteúdo deste arquivo por uma imagem da página assinada pela banca com o comando abaixo:%
% \includepdf{folhadeaprovacao_final.pdf}

\begin{folhadeaprovacao}
	\begin{center}
		{\ABNTEXchapterfont\large\MakeUppercase{\imprimirautor}}
		\vspace*{\fill}\vspace*{\fill}
		\begin{center}
			\ABNTEXchapterfont\Large\MakeUppercase{\imprimirtitulo}
		\end{center}
		\vspace*{\fill}
		\imprimirtextoaprovacao
		\vspace*{1cm}
		\imprimirlocal, 01 de dezembro de 2020.
		\vspace*{\fill}
	\end{center}
	   
	\assinatura{\textbf{\imprimirorientador, Dr.} \\ Orientador\\Instituto Federal de Santa Catarina} 
	\assinatura{\textbf{\imprimircoorientador, Dr.} \\ Coorientador\\Instituto Federal de Santa Catarina }
	\assinatura{\textbf{Professor Membro 1, Me.} \\ Banca Avaliadora\\Instituto Federal de Santa Catarina}
	\assinatura{\textbf{Professor Membro 2, Dr.} \\ Banca Avaliadora\\Instituto Federal de Santa Catarina}
	%\assinatura{\textbf{Professor} \\ Convidado 4}
	\vspace*{1cm}  
	  
\end{folhadeaprovacao}

%---------------------------------------------------------------------%
% Resumos
%---------------------------------------------------------------------%
% Resumo em português
\setlength{\absparsep}{18pt} % ajusta o espaçamento dos parágrafos do resumo
\begin{resumo}
Os exames laboratoriais de sangue, principalmente quando se trata de hemogramas, são um dos tipos de exames mais importantes e mais realizados no âmbito médico. Através dessa prática pode-se descobrir importantes alterações no organismo e é geralmente utilizado como primeiro passo na avaliação da saúde dos pacientes. Embora seja uma prática bastante comum, a sua realização é bastante dificultada nos laboratórios por utilizar um equipamento de maquinário de alto custo de compra e manutenção. Como alternativa a isso, este projeto tem como principal objetivo desenvolver um modelo de \emph{Deep Learning} para realizar esse processo utilizando menos recursos, através da interpretação de imagens de amostras de sangue em placas de Petri para assim ser capaz de elaborar hemogramas de forma automatizada. A partir da elaboração de um mapeamento sistemático da literatura foi possível identificar as principais abordagens de \emph{Deep Learning} (Convolutional Neural Network e Recurrent Neural Network) que podem ser utilizadas para a resolução do problema abordado.

\textbf{Palavras-chave:} Hemograma. Exames de Sangue. \emph{Deep Learning}. Redes Neurais.
\end{resumo}

% Resumo em inglês
\begin{resumo}[Abstract]
	\begin{otherlanguage*}{english}
		This is the english abstract.
		\vspace{\onelineskip}
		\noindent 
		\textbf{Keywords}: latex. abntex. text editoration.
	\end{otherlanguage*}
\end{resumo}

%---------------------------------------------------------------------%
% Inserir lista de ilustrações, tabelas, listagem de códigos, abreviaturas, símbolos
%---------------------------------------------------------------------%
\pdfbookmark[0]{\listfigurename}{lof}
\listoffigures*
\cleardoublepage
% inserir lista de tabelas
\pdfbookmark[0]{\listtablename}{lot}
\listoftables*
\cleardoublepage

%---------------------------------------------------------------------%
% Inserir lista de listings (códigos-fonte)
%---------------------------------------------------------------------%
%\pdfbookmark[0]{\lstlistlistingname}{lol}
%\begin{KeepFromToc}
%\lstlistoflistings
%\end{KeepFromToc}
%\cleardoublepage

%---------------------------------------------------------------------%
% Inserir lista de abreviaturas e siglas
%---------------------------------------------------------------------%
\pdfbookmark[0]{Lista de abreviaturas e siglas}{loa}
% Como usar o pacote acronym
% \ac{acronimo} -- Na primeira vez que for citado o acronimo, o nome completo irá aparecer
%                  seguido do acronimo entre parênteses. Na próxima vez somente o acronimo
%                  irá aparecer. Se usou a opção footnote no pacote, então o nome por extenso
%                  irá aparecer aparecer no rodapé.
%
% \acf{acronimo} -- Para aparecer com nome completo + acronimo.
% \acs{acronimo} -- Para aparecer somente o acronimo.
% \acl{acronimo} -- Nome por extenso somente, sem o acronimo.
% \acp{acronimo} -- Igual o \ac mas deixando no plural com S (inglês).
% \acfp{acronimo}--
% \acsp{acronimo}--
% \aclp{acronimo}--

\chapter*{Lista de abreviaturas e siglas}%
% \addcontentsline{toc}{chapter}{Lista de abreviaturas e siglas}
\markboth{Lista de abreviaturas e siglas}{}

\begin{acronym}
	\acro{RBC}{Red Blood Cells}
	\acro{WBC}{White Blood Cells}
	\acro{CBC}{Complete Blood Count}
	\acro{VCM}{Volume Corpuscular Médio}
	\acro{HCM}{Hemoglobina Corpuscular Média}
	\acro{CHCM}{Concentração de Hemoglobina Corpuscular Média}
	\acro{RDW}{Red Cell Distribution Width}
	\acro{KNN}{K-Nearest Neighbors}
	\acro{ANN}{Artificial Neural Network}
	\acro{DNN}{Deep Neural Network}
	\acro{RNN}{Recurrent Neural Network}
	\acro{CNN}{Convolutional Neural Network}
	\acro{SVM}{Support Vector Machine}
\end{acronym}
\cleardoublepage

%---------------------------------------------------------------------%
% Inserir o sumario
%---------------------------------------------------------------------%
\pdfbookmark[0]{\contentsname}{toc}
\tableofcontents*
\cleardoublepage

% ----------------------------------------------------------
% Elementos Textuais
% ----------------------------------------------------------
\textual

% ----------------------------------------------------------
% Inclusão dos capítulos que estão em outros arquivos .tex
% ----------------------------------------------------------
\chapter{Introdução}
\label{chap:introducao}

A saúde humana sempre foi uma área pilar de toda a sociedade e vem se tornando ainda mais vital para sustentar as demais. Levando em consideração os problemas e situações advindos da pandemia de COVID-19, é necessário pensar em formas de automatizar e auxiliar os profissionais de saúde em suas tarefas, para que consigam focar em problemas mais graves e urgentes. Também com o avanço da tecnologia e dos meios de comunicação, a automação vem se fazendo presente na vida de todos e cada vez mais se torna indispensável nas mais diversas áreas. Para a área da saúde não é diferente, é preciso pensar em formas de, além de automatizar, também facilitar processos cotidianos para assim garantir um foco maior nos problemas mais críticos.

Também como efeito da pandemia, a demanda por exames laboratoriais vem crescendo, e conforme isso acontece, se necessita cada vez mais de profissionais da saúde especializados em atender, analisar e produzir laudos desses exames. Porém nem sempre existe uma equipe suficiente para isso, e então acontece sobrecarga de funções para dar conta dessa demanda.

Esse trabalho tem como principal objetivo buscar maneiras de facilitar e atender a produção de laudos de exames laboratoriais, com um foco em exames de sangue e na produção de hemogramas. De forma que os profissionais da saúde possam utilizar uma ferramenta para auxiliar nesse procedimento. Atualmente, os hemogramas são realizados por máquinas especializadas nessa tarefa e portanto demandam um alto custo financeiro e de manutenção para isso. Esse processo poderia ser facilitado com o uso de algoritmos de \emph{Deep Learning} para a automatização, como forma alternativa ao maquinário especializado.

Os algoritmos de \emph{Deep Learning} (DL) vêm sendo utilizados nas mais diversas áreas, como na medicina \cite{deepLearningMedicine}, na economia \cite{deepLearningEconomy}, nas áreas da educação \cite{deepLearningEducation}, no comércio eletrônico \cite{deepLearningEcommerce} e até em jogos virtuais \cite{deepLearningGaming}. Portanto, DL vem se tornando cada vez mais uma alternativa à métodos tradicionais de realizar tarefas e automatizar processos. Podem ser encontrados alguns trabalhos também na área da saúde, que utilizam técnicas de \emph{Deep Learning} como forma de auxiliar os profissionais em suas tomadas de decisão \cite{deepLearningHealth1} \cite{deepLearningHealth2}.

As técnicas de \emph{Deep Learning} buscam atingir resultados a partir de um grande conjunto de dados. Esses dados devem ser devidamente coletados e adaptados ou seja, pré-processados de forma adequada para a máxima eficiência, dessa forma, um modelo poderá passar por diversas fases de treino, completando o seu treinamento. Com o modelo treinado, pode-se realizar testes com outros dados para obtenção de resultados, que serão pós-processados para uma melhor visualização e apresentados ao profissional da saúde. Todo este processo pode ser chamado de \emph{Knowledge Discovery in Databases} (KDD), que se refere à extração de conhecimento a partir dos dados \cite{kdd} \cite{kdd2}.

Nesse trabalho, busca-se analisar dados de exames de sangue através de imagens de placas de Petri, que são recipientes cilíndricos utilizados pelos profissionais para cultura de microrganismos e análise de materiais \cite{petri}, de forma a elaborar hemogramas e laudos a partir dessas informações. Para isso serão utilizados \emph{datasets} de imagens, a fim de detectar diferentes tipos de células do sangue e chegar em resultados assertivos e úteis para auxiliar também os profissionais da saúde.

% Inicia com uma contextualização, onde se explica como chegou ao problema de pesquisa, e cita-se algumas ideias de trabalhos relacionados. Também deve-se apresentar brevemente a situação atual da área relacionada ao problema que se deseja resolver, conceituar o problema em questão e comentar sobre as técnicas que serão utilizadas para resolver o problema. Tudo isso em no máximo 1 página (3 ou 4 parágrafos).

\section{Problema de Pesquisa}
\label{sec:problema}

Pensando nas formas e aplicações dos algoritmos de \emph{Deep Learning}, presentes nas mais diversas áreas, como um modelo computacional pode ser utilizado para a interpretação de imagens de amostras de sangue em placas de Petri a fim de auxiliar profissionais de laboratório e da saúde na elaboração de laudos científicos e também na sua tomada de decisão?

% Trata-se de uma pergunta, cuja resposta é o seu TCC. Com por exemplo, no TCC do Júlio: “Com base nas técnicas e algoritmos de Machine Learning mais frequentes na Literatura, como ofertar um protótipo com um modelo computacional para predição de evasão escolar a nível de estudante para os cursos de graduação do Instituto Federal de Santa Catarina - Câmpus Caçador?”. Podemos perceber que nosso trabalho é necessariamente uma resposta a uma pergunta. O ideal é que tenha caráter prático, visando a solução de determinado problema.

\section{Hipótese de Pesquisa}
\label{sec:hipotese}

A hipótese para o problema apresentado é que modelos computacionais podem ser treinados para a interpretação de imagens de amostras de sangue em placas de Petri com grande eficiência em prover informações úteis na elaboração automatizada de laudos científicos para profissionais de laboratório e da saúde.

% A hipótese de pesquisa é uma pressuposição sobre o esperado. A observação de uma situação pelo pesquisador, com uma comparação de estudo, dedução lógica da teoria, cultura na qual a problemática é observada, analogias entre duas ou mais variáveis. Normalmente é escrita como uma afirmação associada ao problema de pesquisa. É o que se deseja produzir ao final do TCC.

\section{Objetivos}
\label{sec:objetivos}

\subsection{Objetivo Geral}
Como objetivo geral deste trabalho, deve-se buscar formas de treinamento de um modelo computacional para interpretação de imagens voltado a prover informações úteis sobre hemogramas, possibilitando a geração de laudos científicos automaticamente de forma a auxiliar os profissionais de laboratório e da saúde.

% Converta o seu problema de pesquisa em uma frase afirmativa que inicie com verbo no infinitivo.

\subsection{Objetivos Específicos}
\begin{itemize}
\item Realizar mapeamento sistemático sobre o tema, a fim de identificar as técnicas/algoritmos de \emph{Deep Learning} mais adequados para o reconhecimento de imagens de exames;
\item Buscar dados de imagens de amostras de sangue em bases de dados disponíveis e para esta finalidade;
\item Realizar o pré-processamento dos dados a fim de padronizar e preparar todo o conjunto para o treinamento do modelo computacional;
\item Desenvolver e treinar modelos computacionais de \emph{Deep Learning} a fim de encontrar informações suficientes na análise de amostras de sangue em placas de Petri;
\item Desenvolver um protótipo a partir do modelo computacional pronto e treinado;
\end{itemize}

% Desmembramento do objetivo geral em alguns objetivos específicos, que ao serem atingidos levarão necessariamente ao alcance do objetivo geral.

% Evitar verbos de caráter muito subjetivo, como estudar e conhecer. Preferir utilizar termos como identificar, descrever, propor, entre outros.

\section{Justificativa}
\label{sec:justificativa}
Este estudo busca demonstrar uma forma alternativa de análise das amostras de sangue e na elaboração de laudos, portanto seu principal foco é auxiliar os profissionais da saúde. A contribuição desse estudo poderá ajudar profissionais da saúde a serem mais rápidos em suas decisões sem perder a assertividade, de forma a aumentar a eficiência da análise de exames laboratoriais. Principalmente em momentos de crise, onde a área da saúde é bastante afetada, é necessário ter formas alternativas e associativas em tarefas cotidianas e de extrema importância para a continuidade dos trabalhos. Com esse trabalho, estudiosos da área da computação e também da saúde, poderão ter uma visão muito interessante e associativa de ideias, de forma a auxiliar em novas pesquisas e aplicações.

Outra questão bastante relevante, é em relação aos custos associados, devido ao fato de que o maquinário utilizado hoje para a análise desses exames demanda um custo altíssimo para a sua compra e manutenção. Esse trabalho também possibilitará a análise laboratorial sem a necessidade de compra dessas máquinas caríssimas, de forma a diminuir custos e gastos nesse aspecto.

Embora já existam estudos utilizando \emph{Deep Learning} e também estudos utilizando esses conceitos na área da saúde, esse trabalho tem como principal diferencial trazer a ideia de associar a análise dos modelos de \emph{Deep Learning} com a elaboração de laudos e hemogramas de uma forma automatizada. Logo, se faz necessária a investigação dos conceitos desse trabalho para essa e futuras pesquisas. Este estudo demonstra viabilidade técnica, onde toda a pesquisa e aplicação das definições desse material podem ocorrer durante todo o projeto de trabalho de conclusão de curso. Os livros, artigos e materiais teóricos podem ser providenciados pela instituição e estão disponíveis para o uso.

% Defender a necessidade do estudo, quanto a sua importância, originalidade, oportunidade e viabilidade. Sendo a importância: contribuição do seu estudo na sociedade e na área acadêmica. É importante para quem? Por quê?. A Originalidade: ideia minimamente original. A Oportunidade: período adequado, compatível com as necessidades atuais de conhecimentos e a Viabilidade: existência de recursos necessários para realizar os estudos (tempo, livros, artigos, materiais...)

\section{Organização do texto}
\label{sec:organizacao}

O restante desse trabalho está organizado da seguinte maneira: No \autoref{chap:fund} são apresentados os principais conceitos relacionados a \emph{Deep Learning}, bem como as técnicas estudadas. No \autoref{chap:mapeamento} são apresentados os resultados do mapeamento sistemático da literatura. No \autoref{chap:metodologia} são discutidos os procedimentos metodológicos e no \autoref{chap:cronograma} é apresentado o cronograma para desenvolvimento deste projeto. Por fim, no \autoref{chap:conclusoes} são apresentadas as considerações finais acerca deste trabalho.

% Para finalizar o capítulo introdutório, é interessante que você anuncie ao leitor o conteúdo que ele vai encontrar nos capítulos a seguir. Apresente a informação de que o segundo capítulo é composto pela fundamentação teórica sobre os assuntos X, Y e Z, que o terceiro capítulo trás o mapeamento sistemático, que o quarto capítulo é composto pela metodologia adotada, etc. Normalmente um ou dois parágrafos são suficientes.
\chapter{Fundamentação Teórica}
\label{chap:fund}

Neste capítulo serão apresentados os principais tópicos relacionados ao <Assunto Estudado>, seu conceito e seus impactos na sociedade, bem como as motivações para suas publicações e formas de identificá-las. Além disso, serão abordadas técnicas que permitem <Descrever as técnicas utilizadas>, que serão aplicados para <Tema Proposto>. 

\section{Conceito 1}
\label{sec:conceito1}

Abaixo é apresentada uma figura com o logotipo do Instituto Federal de Santa Catarina. Para inserir uma figura usando o LaTeX, utilizamos a diretiva \emph{figure}. Normalmente referenciamos a figura a partir do seu label, conforme segue. A Figura \ref{fig:exemplo1} mostra o exemplo de uso de imagenos no \LaTeX.

\begin{figure}[!htb]
    \centering
    \caption{Exemplo de uso de imagens no \LaTeX.}
    \includegraphics[width=0.40\textwidth]{img/ifsc.png}
    \legend{Fonte: Elaborada pelo autor.}
    \label{fig:exemplo1}
 \end{figure}
 
 Observe todos os detalhes utilizados. A diretiva \emph{centering} é utilizada para deixar a imagem centralizada. A diretiva \emph{caption} é utilizada para adicionar a legenda na parte superior da imagem. A diretiva \emph{includegraphics} serve para adicionar a imagem propriamente dita, estando neste caso, localizada dentro da pasta \emph{img}. Na mesma diretiva, é possível notar o código \texttt{width=0.40}, que significa que a imagem vai utilizar 40\% da largura do texto. Por fim, a diretiva \emph{legend} é utilizada para indicar a fonte da imagem, e a diretiva \emph{label} para criar uma referência.

\section{Conceito 2}
\label{sec:conceito2}

\section{Conceito 3}
\label{sec:conceito3}

\chapter{Estado da Arte da Área Pesquisada}
\label{chap:mapeamento}

O processo de pesquisa e seleção dos trabalhos relacionados, foi realizado com base em um mapeamento sistemático sobre as pesquisas com propostas para agilizar a identificação e interpretação de análises de sangue. Esta revisão resultou na identificação e seleção dos principais trabalhos de pesquisa no tema deste Projeto de Trabalho de Conclusão de Curso. Outro objetivo deste mapeamento sistemático foi verificar os métodos utilizados para a aplicação de Deep Learning em imagens de sangue em placas de petri de maneira que possam ser aplicados neste projeto de forma satisfatória.

\section{Mapeamento Sistemático da Literatura}

O mapeamento sistemático da literatura é realizado com base na busca e levantamento de artigos, para isso se utiliza uma string de busca para as principais bibliotecas e repositórios de artigos. Esses artigos serão analisados e selecionados conforme a sua área de pesquisa e a sua temática, para inclusão nesse estudo. Para isso, se é utilizado uma ferramenta para automatização dessa tarefa, que é o Parsifal\footnote[1]{https://parsif.al/}, de modo a definir a string de busca, salvar os artigos necessários e realizar a seleção.

As questões de pesquisas levantadas para isso foram, ``Como os algoritmos de Deep Learning podem ser utilizados para a interpretação de exames?'' e ``Como realizar o tratamento de imagens para reconhecimento por modelos de Deep Learning?''. A partir dessas questões se foram extraídas palavras e termos para o direcionamento da pesquisa. Podemos visualizar estas palavras com seus sinônimos na Tabela 1.

\begin{center}
Tabela 1 - Tabela com Palavras-Chave e Sinônimos
\begin{center}
\begin{tabular}{|c|c|}
\hline
\textbf{Palavra-Chave} & \textbf{Sinônimos} \\ \hline
Blood Analysis & Blood Sample \\ \hline
Classification & Interpretation, Recognition \\ \hline
Deep Learning & Artificial Intelligence, Computer Vision, Machine Learning \\ \hline
\end{tabular}
\end{center}
Fonte: Elaborada pelo Autor
\end{center}

Na Tabela 2, é listado as bases de dados onde os artigos foram coletados, a quantidade de cada um de les e a string de busca utilizada na seleção. A mesma string de busca foi utilizado nas três bases de dados, e os artigos encontrados foram dos últimos 5 anos.

\clearpage
\begin{center}
Tabela 2 - Bases de Dados e Quantidade de Artigos Selecionados
\begin{center}
\begin{tabular}{|c|c|c|}
\hline
\textbf{Base de Dados} & \textbf{Artigos} & \textbf{String de Busca} \\ \hline
\multirow{2}{*}{ACM Digital Library} & \multirow{2}{*}{37} & \multirow{6}{*}{\begin{tabular}[c]{@{}c@{}}(``classification'' OR ``interpretation'' OR ``recognition'') AND\\  (``deep learning'' OR ``artificial intelligence'' \\ OR ``computer vision'' OR ``machine learning'') AND\\  (``blood analysis'' OR ``blood sample'')\end{tabular}} \\
 &  &  \\ \cline{1-2}
\multirow{2}{*}{IEEE Digital Library} & \multirow{2}{*}{13} &  \\
 &  &  \\ \cline{1-2}
\multirow{2}{*}{Scopus} & \multirow{2}{*}{114} &  \\
 &  &  \\ \hline
\end{tabular}
\end{center}
Fonte: Elaborada pelo Autor
\end{center}

\subsection{Critérios de Exclusão}

Os artigos coletados na pesquisa através da string de busca, passaram por critérios de exclusão por não se adequarem a esta pesquisa, esses critérios podem ser observados na Tabela 3. 

\begin{center}
Tabela 3 - Critérios de Exclusão
\begin{center}
\begin{tabular}{|c|c|}
\hline
\textbf{Critério de Exclusão} & \textbf{Nº de Artigos Recusados} \\ \hline
O estudo não faz parte da área de pesquisa & 101 \\ \hline
O estudo apresenta resultados fora da computação & 29 \\ \hline
O estudo não é um estudo primário & 6 \\ \hline
O estudo é duplicado & 16 \\ \hline
\end{tabular}
\end{center}
Fonte: Elaborada pelo Autor
\end{center}

A seleção inciou com 164 artigos no total das três bases de dados buscadas. Com a aplicação dos critérios de exclusão, observa-se um resultante de apenas 14 artigos. Isso ocorreu pois 101 artigos foram eliminados no critério ``O estudo não faz parte da área de pesquisa'', que significa que esses artigos tinham alguma relação, porém eram voltados a outras áreas. Outros 29 artigos foram eliminados no critério ``O estudo apresenta resultados fora da computação'', que significa que eram da área de pesquisa, porém com resultados e métodos sem conexão com a computação. Foram também encontrados 6 artigos, que entraram no critério ``O estudo não é um estudo primário'', o que indica que o artigo pode ser uma revisão sistemática da literatura ou semelhante. Por fim, foram eliminados outros 16 artigos por serem duplicados.

\subsection{Critérios de Inclusão}

Os seguintes critérios de inclusão foram definidos:
\begin{itemize}
\item Nova tecnologia para análise de sangue;
\item Processo, método ou técnica para contagem de células sanguíneas;
\item Sistema para elaboração de hemogramas utilizando Deep Learning;
\end{itemize}

Na tabela 4, podemos encontrar todos os 14 artigos selecionados com base nos critérios de inclusão, todos eles se enquadram em pelo menos um deles.

\begin{center}
Tabela 4 - Artigos Selecionados
\begin{center}
\begin{tabular}{|c|l|l|}
\hline
\textbf{ID} & \multicolumn{1}{c|}{\textbf{Título do Artigo}} & \multicolumn{1}{c|}{\textbf{Autores}} \\ \hline
A1 & \begin{tabular}[c]{@{}l@{}}Analyzing microscopic images of \\ peripheral blood smear \\ using deep learning\end{tabular} & \begin{tabular}[c]{@{}l@{}}Mundhra, D. and Cheluvaraju, B. \\ and Rampure, J. and Rai Dastidar, T.\end{tabular} \\ \hline
A2 & \begin{tabular}[c]{@{}l@{}}Automatic detection and classification \\ of leukocytes using \\ convolutional neural networks\end{tabular} & \begin{tabular}[c]{@{}l@{}}Zhao, J. and Zhang, M. \\ and Zhou, Z. and Chu, J. and Cao, F.\end{tabular} \\ \hline
A3 & \begin{tabular}[c]{@{}l@{}}Automatic white blood cell classification \\ using pre-trained deep learning models: \\ ResNet and Inception\end{tabular} & \begin{tabular}[c]{@{}l@{}}Habibzadeh, M. and Jannesari, M. \\ and Rezaei, Z. and Baharvand, H. \\ and Totonchi, M.\end{tabular} \\ \hline
A4 & \begin{tabular}[c]{@{}l@{}}Classification of Human White \\ Blood Cells Using Machine Learning \\ for Stain-Free Imaging \\ Flow Cytometry\end{tabular} & \begin{tabular}[c]{@{}l@{}}Lippeveld, M. and Knill, C. and \\ Ladlow, E. and \\ Fuller, A. and Michaelis, L.J. and \\ Saeys, Y. and Filby, A. and Peralta, D.\end{tabular} \\ \hline
A5 & \begin{tabular}[c]{@{}l@{}}Blood cell classification using the hough\\ transform and \\ convolutional neural networks\end{tabular} & \begin{tabular}[c]{@{}l@{}}Molina-Cabello, M.A. and López-Rubio, E. \\ and Luque-Baena, R.M. and \\ Rodríguez-Espinosa, M.J. and \\ Thurnhofer-Hemsi, K.\end{tabular} \\ \hline
A6 & \begin{tabular}[c]{@{}l@{}}White Blood Cells Image Classification \\ Using Deep Learning with \\ Canonical Correlation Analysis\end{tabular} & Patil, A.M. and Patil, M.D. and Birajdar, G.K. \\ \hline
A7 & \begin{tabular}[c]{@{}l@{}}Image processing and machine learning\\ in the morphological analysis \\ of blood cells\end{tabular} & \begin{tabular}[c]{@{}l@{}}Rodellar, J. and Alférez, S. and Acevedo, A. \\ and Molina, A. and Merino, A.\end{tabular} \\ \hline
A8 & \begin{tabular}[c]{@{}l@{}}Improving blood cells classification in \\ peripheral blood smears using \\ enhanced incremental training\end{tabular} & Al-qudah, R. and Suen, C.Y. \\ \hline
A9 & \begin{tabular}[c]{@{}l@{}}Corruption-Robust Enhancement of \\ Deep Neural Networks\\ for Classification of Peripheral \\ Blood Smear Images\end{tabular} & \begin{tabular}[c]{@{}l@{}}Zhang, S. and Ni, Q. and Li, B. and \\ Jiang, S. and \\ Cai, W. and Chen, H. and Luo, L.\end{tabular} \\ \hline
A10 & \begin{tabular}[c]{@{}l@{}}Convolutional neural network and decision \\ support in medical imaging:\\ Case study of the recognition of \\ blood cell subtypes\end{tabular} & Diouf, D. and Seck, D. and Diop, M. and Ba, A. \\ \hline
A11 & \begin{tabular}[c]{@{}l@{}}Combining Convolutional Neural Network\\ With Recursive Neural Network \\ for Blood Cell Image Classification\end{tabular} & \begin{tabular}[c]{@{}l@{}}Liang, G. and Hong, H. and Xie, W. and\\ Zheng, L.\end{tabular} \\ \hline
A12 & \begin{tabular}[c]{@{}l@{}}Blood diseases detection using \\ classical machine learning algorithms\end{tabular} & Alsheref, F.K. and Gomaa, W.H. \\ \hline
\end{tabular}
\end{center}
Fonte: Elaborada pelo Autor
\end{center}

Todos os artigos selecionados estão relacionados à maneiras e recursos para auxiliar na interpretação de exames de sangue utilizando conceitos de Deep Learning e Machine Learning.

\section{Análise dos trabalhos selecionados}

Por fim, com os artigos selecionados e classificados, é necessário realizar a extração dos dados desses trabalhos, sendo essa a última etapa desse mapeamento sistemático da literatura. É possível perceber que os algoritmos e abordagens mais utilizados são técnicas de \emph{Deep Learning}, como por exemplo, o uso de \emph{Convolutional Neural Network (CNN)} (A1, A2, A3, A4, A5, A6, A8, A9, A10, A11) e de \emph{Recurrent Neural Network (RNN)} (A6, A11), que são abordagens de redes neurais para a classificação das células sanguíneas.

Outros trabalhos utilizam de algoritmos de \emph{Machine Learning} tradicionais para a classificação, como por exemplo, ocorre com o uso de \emph{Random Forest} ou \emph{Decision Trees}  (A2, A4, A7), que são estruturas de árvores de decisão. Também se encontra estudos fazendo uso de \emph{Support Vector Machine (SVM)} (A7) que utilizam vetores de suporte e por fim \emph{K-Means e K-Nearest Neighbors (KNN)} (A12), que faz a classificação levando em conta os vizinhos mais próximos.

\chapter{Procedimentos Metodológicos}
\label{chap:metodologia}

\section{Recursos}

\chapter{Cronograma}
\label{chap:cronograma}

A Tabela \ref{tbl:cronograma} apresenta o cronograma de atividades propostas para o desenvolvimento deste projeto de trabalho de conclusão de curso, de forma a viabilizar <Falar sobre o que se pretende atingir com o projeto>.

\begin{table}[!htb]
\centering
\caption{Cronograma das atividades previstas.}
\label{tbl:cronograma}
\begin{tabular}{|l|c|c|c|c|c|c|c|c|c|c|}
\hline
\multicolumn{1}{|c|}{\textbf{Etapa}}       & \multicolumn{10}{c|}{\textbf{Meses}}                                                                                                                        \\ \hline
                                           & Fev & Mar & Abr & Mai & Jun & Ago & Set & Out & Nov & Dez \\ \hline
Fundamentação Teórica                      & X            & X            &              &              &              &                &                   &               &              &              \\ \hline
\makecell[l]{Mapeamento Sistemático \\ da Literatura}       &              &              & X            & X            &              &                &                   &               &              &              \\ \hline
\makecell[l]{Escrita do Projeto de TCC \\ e Defesa}         &              &              & X            & X            & X            &                &                   &               &              &              \\ \hline
\makecell[l]{Atividade a ser desenvolvida 1}              &              &              &              &              &              & X              &                   &               &              &              \\ \hline
\makecell[l]{Atividade a ser desenvolvida 2}             &              &              &              &              &              &                & X                 &               &              &              \\ \hline
\makecell[l]{Atividade a ser desenvolvida 3} &              &              &              &              &              &                & X                 & X             &              &              \\ \hline
\makecell[l]{Verificação de Aceitação dos \\ Resultados}    &              &              &              &              &              &                &                   & X             &              &              \\ \hline
\makecell[l]{Comparação dos Resultados \\ com a Literatura} &              &              &              &              &              &                &                   & X             & X            &              \\ \hline
Exposição dos Resultados                   &              &              &              &              &              &                &                   &               & X            &              \\ \hline
Escrita do TCC                             &              &              &              &              &              &                &                   &               & X            & X            \\ \hline
Defesa do TCC                              &              &              &              &              &              &                &                   &               &              & X            \\ \hline
\end{tabular}
\vspace{6pt}
\legend{Fonte: Elaborada pelo autor.}
\end{table}

As atividades propostas neste cronograma podem sofrer leves alterações no decorrer do seu desenvolvimento de acordo com a necessidade.

A forma mais fácil de criar tabelas é através de ferramentas gráficas. Geralmente utiliza-se o site \url{https://www.tablesgenerator.com/} para realizar tal atividade, exportando o código LaTeX e colando na parte do texto que ela deve aparecer~\cite{tablegenerator2021}.
\chapter{Conclusões}
\label{chap:conclusoes}

Este projeto iniciou com o interesse de utilizar as técnicas de \emph{Deep Learning} para auxiliar os profissionais da saúde em seus trabalhos, pois com a automatização que essa técnica oferece, outras tarefas podem receber maior atenção. Dentre os diferentes tipo de exames realizados diariamente, o hemograma é o essencial, de forma a ser um ponto de partida para a identificação da maioria das doenças. Portanto, os estudos subsequentes foram direcionados a estudar quais seriam as melhores abordagens para a automatização dessa tarefa.

Foi possível entender todo o processo de produção de hemograma e também encontrar as formas de associar os conceitos de \emph{Deep Learning} nesse processo. Através do treinamento de um modelo computacional inteligente capaz de reconhecer as diferentes células do sangue, desenvolveu-se a contagem de todas os tipos de célula com grande precisão.

Essa abordagem desenvolvida é de grande utilidade, pois permite tirar conclusões sobre o estado atual da saúde do paciente. A contagem das células brancas, vermelhas e também das plaquetas, permitem avaliar se existe algum desequilíbrio em seus níveis, caso apresente algum fator anormal, outros exames mais aprofundados podem ser realizados para verificar estas questões. Entretanto, não se caracteriza como um hemograma completo, pois devido a limitações do \emph{dataset} utilizado, o trabalho também apresentou limitações, principalmente relacionado a classificação das células brancas, onde a contagem foi geral.

Para estudos futuros, é possível utilizar os conceitos aqui abordados para aplicar estes conhecimentos em um conjunto de dados diferente e mais complexo, ou também voltado para outro fim. Outra possibilidade é utilizar métricas de avaliação mais voltadas a essa prática, considerando a área das predições e intersecções com os valores reais. A partir deste resultado, é possível continuar a abordagem adotada por este trabalho para estimar a contagem em relação ao volume e, posteriormente, reproduzir todas as informações presentes em um hemograma.


% ----------------------------------------------------------
% Elementos Pós-Textuais
% ----------------------------------------------------------
\postextual

% ----------------------------------------------------------
% Referências Bibliográficas
% ----------------------------------------------------------
\bibliography{referencias}

%% ----------------------------------------------------------
% Apêndices
% ----------------------------------------------------------
\begin{apendicesenv}
% Imprime uma página indicando o início dos apêndices
%\partapendices

\chapter{Meu primeiro apêndice}
\lipsum[50]

\end{apendicesenv}

% ----------------------------------------------------------
% Anexos
% ----------------------------------------------------------
\begin{anexosenv}
% Imprime uma página indicando o início dos anexos
%\partanexos

\chapter{Meu primeiro assunto de anexo}
\lipsum[30]


\chapter{Segundo assunto que pesquisei}
\lipsum[31]

\end{anexosenv} % anexos (se houver)
\end{document}