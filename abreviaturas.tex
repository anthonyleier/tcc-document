% Como usar o pacote acronym
% \ac{acronimo} -- Na primeira vez que for citado o acronimo, o nome completo irá aparecer
%                  seguido do acronimo entre parênteses. Na próxima vez somente o acronimo
%                  irá aparecer. Se usou a opção footnote no pacote, então o nome por extenso
%                  irá aparecer aparecer no rodapé.
%
% \acf{acronimo} -- Para aparecer com nome completo + acronimo.
% \acs{acronimo} -- Para aparecer somente o acronimo.
% \acl{acronimo} -- Nome por extenso somente, sem o acronimo.
% \acp{acronimo} -- Igual o \ac mas deixando no plural com S (inglês).
% \acfp{acronimo}--
% \acsp{acronimo}--
% \aclp{acronimo}--

\chapter*{Lista de abreviaturas e siglas}%
% \addcontentsline{toc}{chapter}{Lista de abreviaturas e siglas}
\markboth{Lista de abreviaturas e siglas}{}

\begin{acronym}
	\acro{RBC}{Red Blood Cells}
	\acro{WBC}{White Blood Cells}
	\acro{CBC}{Complete Blood Count}
	\acro{VCM}{Volume Corpuscular Médio}
	\acro{HCM}{Hemoglobina Corpuscular Média}
	\acro{CHCM}{Concentração de Hemoglobina Corpuscular Média}
	\acro{RDW}{Red Cell Distribution Width}
	\acro{KNN}{K-Nearest Neighbors}
	\acro{ANN}{Artificial Neural Network}
	\acro{DNN}{Deep Neural Network}
	\acro{RNN}{Recurrent Neural Network}
	\acro{CNN}{Convolutional Neural Network}
	\acro{SVM}{Support Vector Machine}
\end{acronym}